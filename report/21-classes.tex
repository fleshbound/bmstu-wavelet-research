\chapter{Сравнение существующих методов создания текстур посредством генерации шумов}

На основе \cite{survey} осуществлено сравнение методов создания текстур, использующих следующие методы генерации шумов: шум Перлина, анизотропный шум, вейвлет-шум.

В таблице \ref{tab:compare} приведены результаты сравнения, критерии которого расположены по горизонтали и включают в себя следующие:
\begin{enumerate}
	\item требуемый объем памяти (в зависимости от периода шума $N$ и количества измерений пространства $d$);
	\item возможность хранения результатов в виде функций;
	\item возможность создания текстуры, внешний вид которой не зависит от положения точки, на которую она накладывается, в пространстве;
	\item возможность создания текстуры, внешний вид которой зависит от положения точки, на которую она накладывается, в пространстве.
\end{enumerate}

\begin{table}[hbtp]
	\begin{center}
		\begin{flushleft}
			\caption{\label{tab:compare}Сравнение существующих методов создания текстур посредством генерации шумов}
		\end{flushleft}
		\begin{tabular}{|l | l | l | l | l |} 
			\hline 
			~					& {1}	            & {2} & {3} & {4} \\ \hline
			Шум Перлина         & \texttt{$O(N)$}   & Да  & Да  & Нет \\ \hline
			Анизотропный шум    & \texttt{$O(N^d)$} & Нет & Нет & Да  \\ \hline
			Вейвлет-шум         & \texttt{$O(N^d)$} & Да  & Да  & Да  \\ \hline
		\end{tabular}
	\end{center}
\end{table}