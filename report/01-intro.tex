\chapter*{ВВЕДЕНИЕ}
\addcontentsline{toc}{chapter}{ВВЕДЕНИЕ}

За последние 15 лет генерация текстур не перестала быть актуальной задачей компьютерной графики и упоминается в \cite{8942651}, \cite{10042545}, \cite{TRINCHAOANDRADE201228}, \cite{Groueix_2018_CVPR}.
Таким образом, возникает необходимость нахождения методов, которые обеспечивают решение этой задачи.
Существует ряд таких методов, суть которых заключается в использовании генерации шумов (в использовании шумов).

Цель работы --- классификация существующих методов создания текстур посредством генерации шумов.

Для достижения поставленной цели требуется решить следующие задачи:
\begin{enumerate}
	\item проанализировать предметную область создания текстур посредством генерации шумов;
	\item описать основные подходы к решению задачи создания текстур посредством генерации шумов;
	\item сформулировать критерии сравнения применяемых методов;
	\item выполнить сравнение методов в рамках сформулированных критериев.
\end{enumerate}