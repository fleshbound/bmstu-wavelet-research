\chapter*{ВВЕДЕНИЕ}
\addcontentsline{toc}{chapter}{ВВЕДЕНИЕ}

За последние 15 лет генерация текстур не перестала быть актуальной задачей компьютерной графики и упоминается в 
\cite{8942651,10042545,TRINCHAOANDRADE201228,doi:10.1080/15394450902996601,Groueix_2018_CVPR,cabral1993imaging}.
Таким образом, возникает необходимость нахождения методов, которые обеспечивают решение этой задачи.
Существует ряд таких методов, суть которых заключается в использовании генерации шумов (в использовании шумов).

Цель работы --- классификация существующих методов генерации текстур с использованием шумов.

Для достижения поставленной цели требуется решить следующие задачи:
\begin{enumerate}
	\item проанализировать предметную область генерации текстур с использованием шумов;
	\item сформулировать критерии классификации методов генерации текстур с использованием шумов;
	\item описать выбранные методы;
	\item выполнить сравнение выбранных методов в рамках сформулированных критериев.
\end{enumerate}