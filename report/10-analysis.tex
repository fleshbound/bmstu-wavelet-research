\chapter{Обзор вейвлет-шума}

В данном разделе будет рассмотрен вейвлет-шум, суть которого заключается в использовании вейвлетов.

\section{Предварительные замечания}

Вейвлет -- функция независимой переменной, имеющая вид короткой волны (всплеска) \cite{Kryzhevich,Smolentsev,Malla}. Пример вейвлета можно наблюдать на рис. 1.

[рисунок 1 -- пример графика вейвлет функции]

Центр вейвлета -- значение независимой переменной, через которое проходит вертикальная ось симметрии вейвлета.

Пространство вейвлетов $W$ -- набор функций, которые могут быть представлены линейной комбинацией сдвигов и масштабирования одного вейвлета. \cite{Novikov,Meyer}

Масштабирующая функция $\xi$ -- функция, для которой справедливо следующее равенство:

\begin{equation}\label{eq:refine}
	\xi(x)=\sum_{k}p_k\xi(2x-k)
\end{equation}

при условии, что существуют такие числовые коэффициенты $p_k$. \cite{pixar,Novikov}

Согласно \cite{pixar}, если $\phi(x)$ -- функция из $W$, центр которой $x=0$, то любая функция $F(x)$ может быть представлена следующим образом:

\begin{equation}\label{eq:F}
	F(x)=\sum_{i}f_i\phi(x-i)
\end{equation}

где $\phi(x-i)$ -- функция из $W$ с центром $x=i$, $f_i$ -- некоторые числовые коэффициенты.

Пространство $S^0$ -- пространство, состоящее из функций вида \ref{eq:F}.

Функции $G(x)$ вида

\begin{equation}\label{eq:G}
	G(x)=\sum_{i}g_i\phi(2x-i)
\end{equation}

аналогично составляют пространство $S^1$. \cite{pixar}

Если рассматриваемая в \ref{eq:G}, \ref{eq:F} функция $\phi$ является масштабирующей, то $S^1$ расширяет $S^0$, то есть включает все функции $S^0$. \cite{pixar,Novikov}

Вейвлет-анализ -- процесс определения того, расширяет ли пространство $S^1$ пространство $S^0$. \cite{pixar}.

Тогда функция $F(x)$ из \ref{eq:F} представляется с помощью коэффициентов $f_i^{\uparrow}$ в пространстве $S^1$:

\begin{equation}\label{eq:Fup}
	F(x)=\sum_{i}f_i^{\uparrow}\phi(2x-i)
\end{equation}

Коэффициенты $f_i^{\uparrow}$ выражаются с использованием формулы \ref{eq:refine}:

\begin{equation}\label{eq:f_iup}
	f_i^{\uparrow}=\sum_{k}p_{i-2k}f_k
\end{equation}
%Повышение разрешения функции $F(x)$ (вдвое) -- вычисление коэффициентов $f_i^{\uparrow}$.

Согласно \cite{pixar}, коэффициенты $g_i^{\downarrow}$, необходимые для представления функции $G(x)$ из \ref{eq:G} в пространстве $S^0$, вычисляются с помощью вейвлет-анализа и являются равными

\begin{equation}\label{eq:g_idown}
	g_i^{\downarrow}=\sum_{k}a_{k-2i}g_k
\end{equation}

а функция $G(x)$ представляется как

\begin{equation}\label{eq:G_divide}
	G(x)=G^{\downarrow}(x) + D(x)
\end{equation}

где $D(x)$ -- функция из $S^1$, которая не может быть выражена в $S^0$ через какие-либо коэффициенты.
%Понижение разрешения функции $G(x)$ (вдвое) -- вычисление коэффициентов $g_i^{\downarrow}$.

\section{Описание метода}

Изображение $X$ представляется набором числовых коэффициентов $(\dots,x_i,\dots)$:

\begin{equation}\label{eq:x_image}
	X=(\dots,x_i,\dots)
\end{equation}

Входные данные: функция $B(x)$ из $W$, изображение $R$.

Выходные данные: изображение $N$.

Авторы \cite{pixar} представляют алгоритм вейвлет-шума тремя семантическими шагами:

\begin{enumerate}
	\item получение $R(x)$ (в пространстве $S^1$);
	\item получение $R^\downarrow(x)$ (в пространстве $S^0$);
	\item получение $R^{\downarrow\uparrow}(x)$ (в пространстве $S^1$);
	\item получение $N(x)$ (в пространстве $S^1$).
\end{enumerate}

При заданном $x$ вычисляется значение функции $N(x)$ и получается изображение $N=(\dots,n_i,\dots)$.

Пример визуализации получения результата алгоритма можно наблюдать на рис. 2.

[рисунок 2 -- пример визуализации получения из $R$ $N$ \cite{pixar}]

Рассмотрим представленные шаги более подробно, основываясь на информации из \cite{pixar}.

\subsection{Получение $R^\downarrow$}

Согласно формуле \ref{eq:x_image} изображение $R$ представляется следующим образом:

\begin{equation}\label{eq:R_image}
	R=(\dots,r_i,\dots)
\end{equation}

Аналогично формуле \ref{eq:G} выражается $R(x)$:

\begin{equation}\label{eq:R}
	R=\sum_{i}r_i B(2x-i)
\end{equation}

\subsection{Получение $R^\downarrow$}

Согласно формуле \ref{eq:G_divide}:

\begin{equation}\label{eq:R_divide}
	R(x)=R^{\downarrow}(x)+N(x)
\end{equation}

$N(x)$ выражается из формулы \ref{eq:R_divide} как

\begin{equation}\label{eq:N_divide}
	N(x)=R(x)-R^{\downarrow}(x)
\end{equation}

С помощью формулы \ref{eq:F} получаем

\begin{equation}\label{eq:Rdown}
	R^{\downarrow}(x)=\sum_{i}r_{i}^{\downarrow}B(x-i)
\end{equation}

где $r_{i}^{\downarrow}$ выражаются с помощью формулы \ref{eq:g_idown}:

\begin{equation}\label{eq:r_idown}
	r_i^{\downarrow}=\sum_{k}a_{k-2i}r_k
\end{equation}

\subsection{Получение $R^{\downarrow\uparrow}$}

C помощью формул \ref{eq:R} и \ref{eq:Rdown} выражение \ref{eq:N_divide} преобразовывается как

\begin{equation}\label{eq:N_1}
	N(x)=\sum_{i}r_{i}B(2x-i)-\sum_{i}r_{i}^{\downarrow}B(x-i)
\end{equation}

С учетом формулы \ref{eq:Fup} и \ref{eq:Rdown} получается выражение

\begin{equation}\label{eq:Rdownup}
	R^{\downarrow\uparrow}(x)=\sum_{i}r_{i}^{\downarrow\uparrow}B(2x-i)
\end{equation}

где коэффициенты $r_{i}^{\downarrow\uparrow}$ выражаются с учетом формулы \ref{eq:f_iup}:

\begin{equation}\label{eq:r_idownup}
	r_i^{\downarrow\uparrow}=\sum_{k}p_{i-2k}r_k^{\downarrow}
\end{equation}

\subsection{Получение $N$}

С использованием формулы \ref{eq:Rdownup} выражение \ref{eq:N_1} записывается как

\begin{equation}\label{eq:N}
	N(x)=\sum_{i}r_{i}B(2x-i)-\sum_{i}r_{i}^{\downarrow\uparrow}B(2x-i)=\sum_{i}n'_{i}B(2x-i)
\end{equation}

где $n'_i=r_i-r_i^{\downarrow\uparrow}$.