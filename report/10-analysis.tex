\chapter{Анализ предметной области}

%Текстура --- стиль заполнения, имитирующий сложную рельефную объемную плоскость, выполненную из какого-то материала \cite{Porev}.

Стандартной задачей компьютерной графики является визуализация изображений, способы осуществления которой делятся на растровый и векторный \cite{Porev}, \cite{boreskov}.
При использовании растрового способа изображение $X$ представляется в виде набора точек (пикселей) \cite{Porev}, \cite{boreskov}:
\begin{equation}\label{eq:image_x}
	X=(x_1,\dots,x_n),
\end{equation}
где $x_i$ --- целое число, равное значению одной из RGB-компонент цвета пикселя.

Текстура --- растровое изображение, используемое в компьютерной графике для изменения внешнего вида поверхностей и других объектов без изменения их формы \cite{smorkalov}.

Псевдослучайный метод получения чисел (точек) --- метод получения случайных чисел (точек) с возможностью повторного воспроизведения набора при одних и тех же входных данных \cite{мухамеджанов2018генератор}.

Шум --- набор точек, полученный псевдослучайным методом \cite{petrova2022}.

Таким образом, одним из способов создания текстур является генерация шумов, результатом которой является набор точек или функций, к которым требуется применить операции для получения необходимых значений.

Согласно \cite{survey}, для сравнения методов создания текстур с использованием генерации шумов выделяются следующие критерии:
\begin{enumerate}
	\item требуемый объем памяти;
	\item возможность хранения результатов в виде функций;
	\item возможность создания текстуры, внешний вид которой не зависит от положения точки, на которую она накладывается, в пространстве;
	\item возможность создания текстуры, внешний вид которой зависит от положения точки, на которую она накладывается, в пространстве.
\end{enumerate}

Далее будут рассмотрены существующие решения задачи --- методы генерации шумов, которые лежат в основе соответствующих методов создания текстур: шум Перлина, анизотропный шум, вейвлет-шум.