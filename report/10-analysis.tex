\chapter{Анализ предметной области}

%Текстура --- стиль заполнения, имитирующий сложную рельефную объемную плоскость, выполненную из какого-то материала \cite{Porev}.

Стандартной задачей компьютерной графики является визуализация изображений, способы осуществления которой делятся на растровый и векторный \cite{Porev}, \cite{boreskov}.
При использовании растрового способа визуализации изображение $X$ представляется в виде набора точек \cite{boreskov}, \cite{Porev}:
\begin{equation}\label{eq:image_x}
	X=(\dots,x_i,\dots)
\end{equation}

Текстура --- растровое изображение, используемое в компьютерной графике для наложения \cite{smorkalov}.

Псевдослучайный метод получения чисел (точек) --- метод получения случайных чисел (точек) с возможностью повтора набора при повторе входных данных \cite{мухамеджанов2018генератор}.

Шум --- набор точек, полученный псевдослучайным методом \cite{petrova2022}.

Согласно \cite{}, для сравнения методов генерации шумов можно выделить следующие критерии:
\begin{enumerate}
	\item 
\end{enumerate}