\chapter*{Введение}
\addcontentsline{toc}{chapter}{Введение}

В данной лабораторной работе рассматриваются алгоритмы поиска расстояний Левенштейна и Дамерау-Левенштейна.

Расстояние Левенштейна --- это минимальное число редакционных операций (вставка, удаление, замена), необходимых для преобразования первой строки во вторую. Расстояние Дамерау-Левеншейна является модификацией расстояния Левенштейна, которая отличается добавлением в список операций редактирования операции перестановки.



Целью данной лабораторной работы является описание и исследование алгоритмов поиска расстояний Левенштейна и Дамерау-Левенштейна.

Для поставленной цели необходимо выполнить следующие задачи:
\begin{enumerate}[label=---]
	\item изучить алгоритмы поиска расстояний Левенштейна и Дамерау-Ле-\\венштейна (рекурсивный и с мемоизацией);
	\item реализовать указанные алгоритмы;
	\item выбрать инструменты для замера процессорного времени выполнения алгоритмов;
	\item провести анализ затрат реализаций алгоритмов по времени и по памяти, определить влияющие на них характеристики.
\end{enumerate}